\documentclass[a4paper,10pt]{scrartcl}
\usepackage{babel}

\usepackage{scrlayer-scrpage}
\lohead{Aufgabe 3: Telepaartie}
\rohead{Team-ID: ***REMOVED***}
\cfoot*{\thepage{}}

\usepackage{listings}
\usepackage{color}
\definecolor{mygreen}{rgb}{0,0.6,0}
\definecolor{mygray}{rgb}{0.5,0.5,0.5}
\definecolor{mymauve}{rgb}{0.58,0,0.82}
\lstset{
  keywordstyle=\color{blue},
  commentstyle=\color{mygreen},
  stringstyle=\color{mymauve},
  rulecolor=\color{black},
  basicstyle=\footnotesize\ttfamily,
  numberstyle=\tiny\color{mygray},
  numbers=left,
  literate=%
  {Ö}{{\"O}}1
  {Ä}{{\"A}}1
  {Ü}{{\"U}}1
  {ß}{{\ss}}1
  {ü}{{\"u}}1
  {ä}{{\"a}}1
  {ö}{{\"o}}1
}

\title{Aufgabe 3: Telepaartie}
\author{Team-ID: ***REMOVED*** \\\\
	    Team-Name: ***REMOVED*** \\\\
	    Bearbeiter dieser Aufgabe: \\
	    ***REMOVED***\\\\}
\date{\today}

\begin{document}

\maketitle
\tableofcontents

\section{Lösungsidee}
??

\section{Umsetzung}
??

\section{Beispiele}
Hier werde ich die Beispiele von der BWINF-Webseite und die Besispiele für aus der Aufgabe für die erweiterte Version aufführen.

\subsection{Beispiel 1}
Verteilung: (2, 4, 7)
\begin{lstlisting}
[2, 4, 7]
  1 <> 3
[4, 4, 5]

[4, 4, 5]
  1 <> 2
[0, 8, 5]

2 Schritte
\end{lstlisting}

\subsection{Beispiel 2}
Verteilung: (3, 5, 7)
\begin{lstlisting}
[3, 5, 7]
  1 <> 2
[6, 2, 7]

[6, 2, 7]
  1 <> 2
[4, 4, 7]

[4, 4, 7]
  1 <> 2
[0, 8, 7]

3 Schritte
\end{lstlisting}

\subsection{Beispiel 3}
Verteilung: (80, 64, 32)
\begin{lstlisting}
[80, 64, 32]
  1 <--> 3
[48, 64, 64]

[48, 64, 64]
  2 <--> 3
[48, 0, 128]

2 Schritte
\end{lstlisting}

\subsection{Beispiel 4}
L(10) = 2

\subsection{Beispiel 5}
L(100) = 7

\section{Quellcode}
\lstinputlisting[language=python]{"Aufgabe 3: Telepaartie/telepaartie.py"}

\end{document}
